\section{Implementazione}
\definecolor{codegreen}{rgb}{0,0.6,0}
\definecolor{codegray}{rgb}{0.5,0.5,0.5}
\definecolor{codepurple}{rgb}{0.58,0,0.82}
\definecolor{backcolour}{rgb}{0.95,0.95,0.92}

In questa sezione verranno mostrati i frammenti più importanti del codice
\lstdefinestyle{mystyle}{
    backgroundcolor=\color{backcolour},   
    commentstyle=\color{codegreen},
    keywordstyle=\color{magenta},
    numberstyle=\tiny\color{codegray},
    stringstyle=\color{codepurple},
    basicstyle=\ttfamily\footnotesize,
    breakatwhitespace=false,         
    breaklines=true,                 
    captionpos=b,                    
    keepspaces=true,                 
    numbers=left,                    
    numbersep=5pt,                  
    showspaces=false,                
    showstringspaces=false,
    showtabs=false,                  
    tabsize=2
}
\lstset{style=mystyle}
\subsection{Avvio del DB NoSQL}
\lstinputlisting[language=bash]{./Code/avvio.sh}

Automaticamente il controllore mettendo i suoi dati (validi) apre la connessione al database e va direttamente alla pagina di controllo degli abbonamenti
\subsection{Creazione del Database}
\lstinputlisting[language=bash]{./Code/CreazioneDB.sh}

La creazione del database viene fatta una sola volta, prima del rilascio dell'applicazione. 
\\Potrebbero succedere problemi hardware, oppure il link potrebbe perdere la connessione magari durante un processamento dei dati, per questo una soluzione è quella di avere un backup disponibile appena succede un problema, quindi i dati vengono replicati per assicurare che non ci siano fallimenti. Cassandra posiziona le repliche dei dati su nodi diversi in base a due fattori:
\paragraph{•} Strategia di replicazione
\paragraph{•} Fattore di replicazione
Il primo dice dove posizionare le replica, il secondo invece determina quante repliche vengono posizionate, un esempio lo troviamo nella riga 4 dove abbiamo la stringa \textit{'replication\_factor' : 3}, questo indica che vengono fatte 3 repliche su 3 nodi diversi. Solitamente per garantire che non ci siano fallimenti il fattore di replicazione deve essere 3.
\\Nella riga 3 vediamo la stringa \textit{'SimplyStrategy'}, questa viene usata quando abbiamo un solo data center, quindi la prima replica viene posizionata sul nodo selezionato dal partizionatore, dopo questo, le restanti repliche vengono posizionate in senso orario a partire dalla direzione del nodo.
\subsection{Creazione tabella DB}
\lstinputlisting[language=bash]{./Code/CreazioneTabella.sh}

Anche questa parte di codice, come la creazione del database viene fatta una sola volta, prima del lancio dell'applicazione
\subsection{Connessione al DB NoSQL}
\lstinputlisting[language=Java]{./Code/Connessione.java}

In questo frammento di codice viene mostrato il metodo con il quale viene fatta la connessione al database NoSQL Cassandra!
\subsection{Apertura della connessione con la porta seriale in Java}
\lstinputlisting[language=Java]{./Code/AperturaSeriale.java}

Nel frammento di codice appena visto c'è la dichiarazione delle porte che si vanno ad utilizzare, il \textbf{DATA\_RATE}, ovvero la quantità di dati digitali che possono essere trasferiti su un canale in un determinato intervallo temporale e l'apertura della connessione tramite la funzione \textbf{portId.open}
\subsection{Rilevamento presenza tag NFC Arduino}
\lstinputlisting[language=c++]{./Code/RilevamentoPresenze.cpp}

In questo frammento di codice viene evidenziato come il chip NFC se presente viene scannerizzato e l'unica cosa che verrà presa sarà il payload e non l'intestazione! 
\subsection{Creazione abbonamento in Java}
\lstinputlisting[language=Java]{./Code/CreazioneAbbonamento.java}

\subsection{Scrittura abbonamento su tag NFC }
\subsection{Scrittura nel DB dei dati}
