\section{Implementazione}

In questa sezione verranno sottolineate i frammenti più importanti del codice.
\subsection{Apertura della Connessione con Java}
\begin{lstlisting}[language=Java]
private static final String PORT_NAMES[] = { 
			"/dev/cu.usbmodem143301", // Mac OS X
			"COM3", // Windows
	};
private static final int DATA_RATE = 9600;

serialPort = (SerialPort) portId.open(this.getClass().getName(),
					TIME_OUT);

\end{lstlisting}
Nel frammento di codice appena visto c'è la dichiarazione delle porte che si vanno ad utilizzare, il \textbf{DATA\_RATE}, ovvero la quantità di dati digitali che possono essere trasferiti su un canale in un determinato intervallo temporale e l'apertura della connessione tramite la funzione \textbf{portId.open}

\subsection{Rilevazione presenza tag NFC Arduino}
\begin{lstlisting}[language=c++]
      NfcTag tag = nfc.read(); 
      if(tag.hasNdefMessage())
      {[...]    
          NdefRecord record = message.getRecord(i);
          int payloadLength = record.getPayloadLength();
          byte payload[payloadLength];
          record.getPayload(payload);
          Serial.write(payload,payloadLength);
        [...]}
\end{lstlisting}
In questo frammento di codice viene evidenziato come il chip NFC se presente viene scannerizzato e l'unica cosa che verrà presa sarà il payload e non l'intestazione! 
\subsection{Scrittura su tag NFC}
\subsection{Connessione al DB NoSQL}
\begin{lstlisting}[language=Java]
public void connect(final String node, final int port) {
		CodecRegistry codecRegistry = new CodecRegistry();
		codecRegistry.register(new DateCodec(TypeCodec.date(),
		 Date.class));
		cluster = Cluster.builder().addContactPoint(node).
		withPort(port).withCodecRegistry(codecRegistry).build();
 
		final Metadata metadata = cluster.getMetadata();
	 [...]
		session = cluster.connect();
	}
\end{lstlisting}
In questo frammento di codice viene mostrato il metodo con il quale viene fatta la connessione al database NoSQL Cassandra!
\subsection{Avvio di Cassandra}
\begin{lstlisting}[language=C]
 ~ % source ~/.bash_profile

~ % cassandra -f 
\end{lstlisting}
Accedendo al terminale vengono lanciati questi comandi per l'avvio del processo di \textbf{Cassandra}