\section{Introduzione}

\subsection{Idea}

L'idea di base è nata, durante un viaggio nei paesi nordici quando vidi che salendo sui mezzi di trasporto pubblici, le persone utilizzavano delle tessere magnetiche, analizzando e chiedendo scoprii poi essere tessere con all'interno dei chip NFC, andando avanti negli anni scoprii che in altri paesi, oltre all'utilizzo delle tessere in parallelo venivano usati anche gli smartphone con in chip direttamente incluso nel telefono.
\\Da li ho preso spunto chiedendomi "Perché non portare anche nel nostro paese una tecnologia del genere?" .l'idea di fondo è quindi quella di avere un abbonamento sempre a portata di mano, facilmente rinnovabile, meno ingombrante, e anche più sicuro, inoltre è anche un'idea \textbf{eco-friendly}, infatti si può pensare che al posto di comprare un biglietto od un carnet di biglietti, per poi buttarli via dopo l'utilizzo, si ha a disposizione una tessera magnetica nella quale viene caricato il biglietto/carnet, e dopo l'utilizzo basterà semplicemente rinnovarlo o cambiarne la tipologie, evitando così uno spreco di carta.

\subsection{MyNBS}

MyNBS (My NFC Bus Subscription) è un'applicazione sviluppata in Java con un'interfaccia grafica che permette la sottoscrizione di un abbonamento per i pullman o il controllo di un abbonamento già esistente. Abbiamo quindi due funzionalità che vanno a dividersi in molteplici step: 
\\ \textbf{Sottoscrizione abbonamento}:
\paragraph{1.} L'operatore inserisce i dati dell'utente e le zone volute per l'abbonamento in un'interfaccia grafica
\paragraph{2.} L'operatore posizione sull'antenna NFC il tag, nel quale verranno scritti in maniera codificata i dati dell'utente.
\bigskip
\\\textbf{Controllo abbonamento}:
\paragraph{1.} Il controllore seleziona la/e zona/e dove è in questo momento
\paragraph{2.} Posizione sopra il lettore il tag NFC, sia che l'abbonamento è valido per quella zona che non è valido verrà segnalato, nel secondo caso verranno evidenziate le zone di validità dell'abbonamento.

\subsection{Obiettivi}
L'obiettivo principale è quello di avere con se un abbonamento facilmente trasportabile, poco ingombrante e sicuro, infatti tramite la tecnologia NFC e l'implementazione della crittografia è possibile avere un'autenticazione sicura ed evitare anche che qualcuno di esterno riesca ad interpretare i dati del chip anche se dovesse riuscire a copiarlo. Inoltre per le forze dell'ordine è molto più semplice controllare quel chip e i dati associati piuttosto che dover guardare una carta che andando avanti nel tempo subirebbe l'usura e risulterebbe quindi di difficile comprensione.